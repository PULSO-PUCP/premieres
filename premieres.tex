\documentclass[a4paper, 12pt]{article}
\usepackage{authblk}
\renewcommand{\Affilfont}{\footnotesize}
\title{Cabinet under Pressure: Survival Analysis of Peru’s Prime Ministers Since 1980}
\author[1,2]{ Jose Manuel Magallanes \thanks{The authors would like to thank the research assistants from PULSO-PUCP: Alexandra Porras, Alfredo Aro, Romina Loayza, and Ivana Delgado, for their support  and decication to this work.}}
\affil[1]{PULSO -Institute of Social Analytics and Strategic Intelligence and Department of Social Sciences, Pontificia Universidad Catolica del Peru, San Miguel 15088, Lima, Peru}
\affil[2]{University of Massachusetts-Amherst; University of Washington -Seattle; and Universidad Nacional Mayor de San Marcos-Lima}
\affil[*]{Corresponding author: jmagallanes@pucp.edu.pe}


\date{\today}  %% manually: \date{March 6, 2025} 
\usepackage[natbibapa]{apacite} %% for bibliography
\usepackage{rotating, graphicx} %% for rotating tables
\usepackage{adjustbox} % size of plots and tables
\usepackage{chngcntr}% section numbering
\usepackage{amssymb}
\counterwithin{table}{section}\counterwithin{figure}{section}
\usepackage{Sweave}
\begin{document} % every "begin: needs and "end"
\Sconcordance{concordance:premieres.tex:premieres.Rnw:1 21 1 1 0 138 1 1 24 18 0 1 2 %
179 1 1 5 11 0 1 2 270 1}

\maketitle 
\begin{abstract}
This study examines the political durability of Peru’s Prime Ministers (Presidente del Consejo de Ministros, PCM) since the country’s democratic return in 1980. Using survival analysis, we explore how political context, institutional conditions, and crisis dynamics shape the tenure of these key presidential appointees. Through a Cox proportional hazards model, we test the influence of presidential popularity, legislative fragmentation, cabinet reshuffles, and regime instability on the risk of early dismissal or resignation. The results illuminate how informal power-sharing and institutional fragility influence executive coordination in a hyper-presidential regime.
\end{abstract}


\section*{Introduction} % * to unnumber

The median tenure of a Peruvian Prime Minister (PCM) since 1980 is 254 days (around 8 months), yet variation is striking. \'{A}ntero Flores\-Ar\'{a}oz lasted only six days in November 2020, whereas Manuel Ulloa served 864 days in the early 1980s and Jorge del Castillo survived 805 days during Alan Garc\'{i}a\'s second term. Dispersion persists even within a single presidential period: three PCMs held the post during Pedro Castillo\'s first five months, while Alberto Ot\'{a}rola remained in office for 440 days under Dina Boluarte. This inconsistency under ostensibly similar institutional rules raises a central puzzle: Which political, institutional, and situational factors condition the durability of Peru\'s Prime Ministers?

Although the Constitution describes the PCM merely as the president’s first minister—responsible for countersigning decrees and coordinating the cabinet—practice since 1980 indicates that the office routinely performs head-of-government functions: drafting the policy agenda, negotiating confidence votes, and representing the executive in congressional interpellations. The PCM thus acts as the president’s chief political shield. During normal times, the office facilitates legislative compromise and signals technocratic competence; in crises, it functions as a “circuit-breaker,” a high-visibility scapegoat whose dismissal absorbs congressional anger and preserves presidential tenure. Analysing the determinants of PCM survival therefore illuminates how Peru’s hyper-presidential system manages accountability, blame assignment, and policy coordination in the absence of strong parties.



\section{Background}\label{backg-tables} % label for crossref

Legal foundations (Art.124, 1993 Constitution): The PCM is named by the President but becomes fully operative only after the Council of Ministers receives a congressional vote of confidence (Art.130). Article 124 establishes five formal powers that lift the office above an ordinary minister: (1) countersigning every presidential decree and law, thereby granting legal validity; (2) convening and presiding over the Council of Ministers when the President is absent; (3) coordinating sectoral policies and monitoring implementation across ministries; (4) presenting the government’s general policy and annual budget bill to Congress; and (5) proposing—on behalf of the Council—votes of confidence or requesting the President to dissolve Congress after two denials. Taken together, these clauses make the PCM the linchpin between executive decree authority and legislative oversight, subject to individual censure (Art.132) and collective cabinet responsibility. This constitutional design explains why dismissing the PCM can defuse political crises without toppling the President.
Informal evolution: From symbolic coordinator to key political operator. In the early 1980s the PCM was largely a symbolic co‑signatory—Manuel Ulloa chaired cabinet meetings but real bargaining occurred directly between President Belaúnde and party caucuses. Two changes transformed the post:
Fujimori’s executive re‑engineering (1990‑2000): frequent decree lawmaking and the 1992 Constitution concentrated agenda control in the Palace. The PCM became the de facto gatekeeper of emergency decrees, crisis communication, and IMF negotiations. Guillermo Larco Cox’s 1989 and Hurtado Miller’s 1990 tenures were early signs; by 1995 Dante Córdova was publicly branded “Premier”, marking a rhetorical shift from coordinator to operator.

Confidence‑vote politics after 2000: Article 130 requires each new Council to obtain a congressional vote of confidence, but presidents have repeatedly weaponised the PCM to reset relations with an antagonistic Congress—e.g., Salomón Lerner (2011), César Villanueva (2013, 2018), Pedro Cateriano(2020). The PCM now drafts policy speeches, counts votes, and negotiates floor time, acting as a quasi‑prime minister in a party‑fragmented legislature.

Consequently, career technocrats and seasoned politicians alike view the PCM as the highest non‑presidential prize—yet also the first political “fuse” to blow when scandals erupt. Our duration data confirm this duality: the office’s power has grown, but so has its volatility, illustrating the gap between formal rules and informal political practice.

Role during key periods (see Table \ref{durationPerEra}):

\begin{enumerate}
    \item Democratic transition (1980–1989): Under Belaúnde and García I, the PCM was chiefly a coalition‑balancing broker: Manuel Ulloa, Fernando Schwalb, and Luis Alva Castro used the office to mediate between the president and newly elected party caucuses, but limited decree use kept the post largely symbolic. Average tenure was around 400 days.
    \item Fujimori era (1990–2000): Executive decree authority and the 1992 Constitution turned the PCM into the president’s operational shield. PMs like Hurtado Miller (199 days) and Pandolfi (792 days across two spells) managed IMF talks and emergency legislation, yet were dismissed when inflation, corruption, or congressional frictions peaked, thus underscoring the role as sacrificial buffer. Average tenure was around 790 days, and presented the smallest variability in these periods.
    \item Post‑2000 democratic turbulence: With weak parties and fragmented congresses, the PCM became the pivot for confidence‑vote politics. Thirty‑four different PMs have served since 2001, median tenure just 232 days. Presidents Toledo, Humala, Vizcarra, Castillo, and Boluarte each used rapid PCM turnover to re‑negotiate legislative support—e.g., Cateriano’s 22‑day tenure (2020) followed by Martos’s 96 days as the executive searched for votes of confidence. Average tenure was around 805 days, and presented the highest variability in these periods.
\end{enumerate}



\begin{Schunk}
\begin{Sinput}
> library(dplyr)
> library(kableExtra)
> library(knitr)
> linkGit1='https://github.com/PULSO-PUCP/premieres'
> linkGit2='/raw/refs/heads/main/primes.xlsx'
> linkPrimes=paste0(linkGit1,linkGit2)
> data=rio::import(linkPrimes)
> namesHeaders=c("Era","count","mean","min",'max','range')
> summary_data <- data %>%
+   group_by(era) %>%
+   summarise(
+     N = n(),
+     Mean_Var1 = mean(days_duration),
+     Min = min(days_duration),
+     Max = max(days_duration),
+     Range=max(days_duration)-min(days_duration)
+   ) %>%
+      kable(format = "latex", 
+            digits = 2,                #label cross-ref!!
+            caption = "Duration in Days by era\\label{durationPerEra}",
+            col.names = namesHeaders)%>%
+     kable_styling(full_width = F,latex_options = "h")
> summary_data
\end{Sinput}
\begin{table}[!h]
\centering
\caption{Duration in Days by era\label{durationPerEra}}
\centering
\begin{tabular}[t]{l|r|r|r|r|r}
\hline
Era & count & mean & min & max & range\\
\hline
A.PreFujimori & 9 & 405.44 & 138 & 864 & 726\\
\hline
B.Fujimori & 13 & 302.08 & 78 & 792 & 714\\
\hline
C.PostFujimori & 35 & 256.77 & 6 & 805 & 799\\
\hline
D.before\_1980 & 167 & 262.43 & 1 & 1822 & 1821\\
\hline
\end{tabular}
\end{table}\end{Schunk}


\section{Literature Review}\label{letrev}

How is the  survival of Prime ministers in semi-presidential or similar regimes? Empirical work shows that the strategic logic of coalition maintenance and partisan bargaining strongly shapes how long a prime‑minister‑type appointee lasts. \citet{amorim_neto_presidential_2006}, analysing more than 1,000 minister spells in ten semi‑presidential countries (1972–1999), finds that (i) prime ministers who belong to the president’s party or to pivotal coalition partners face significantly lower hazards of exit, and (ii) periods of cohabitation sharply increase replacement risk because presidents have incentives to reshuffle PMs allied with the rival camp. \citet{cheibub_government_2004} compare cabinet durability across regime types and show that survival is shortest under pure presidentialism, longest under parliamentarism, with semi‑presidential systems in between; within the latter, minority presidents and higher party fragmentation elevate the hazard of PM turnover. These results highlight three mechanisms—partisan alignment, coalition breadth, and executive – legislative congruence—that our Peruvian PCM analysis will explicitly test.

Is there Cabinet instability in Peru? Research converges on three reinforcing explanations. 
\citet{levitsky_latin_1999} shows that the collapse of the party system during the 1990s replaced partisan accountability with presidential patronage, turning ministers into expendable crisis‑management tools. Successive presidents have alternated between appointing party heavyweights and high-profile technocrats as Prime Minister in hopes of stabilising cabinet politics, yet neither strategy has reliably eased the post-Fujimori deadlock created by Peru’s fragmented Congress and wary executive—a pattern documented by Tanaka’s diagnosis of a “democracia sin partidos,” Valladares’s account of benign but paralysing legislative oversight, and Dargent’s portrait of technocrats who operate with limited political backing \citep{tanaka_democracia_2005,valladares_horror_2012, dargent_technocracy_2014}. Collectively, these studies depict cabinet instability as a product of weak party institutionalization, contentious executive–legislative relations, and recurrent political crises—mechanisms our survival analysis of PCM tenure will test.

Is there a role from Informal institutions and the elite in Peru?\citet{helmke_informal_2004} define informal institutions as socially shared, unwritten rules that structure political behaviour and interact with formal rules in four distinct ways—complementary, accommodating, competing, and substitutive. In much of Latin America, competing and substitutive informal institutions are common and often undermine formal cabinet‑appointment procedures. Presidents rely on personalised patron‑client networks, strategic scapegoating norms, and informal blame‑shifting rituals to manage crises, which accelerates elite turnover. In Peru, where party institutionalisation is weak, an unwritten convention has emerged: when legislative censure or scandal threatens presidential standing, the Prime Minister is sacrificed as a cost‑containing signal of accountability, enabling the president to reset relations with Congress without relinquishing core power. This suggests that PCM survival hinges not only on formal constitutional design but also on the strength and interaction of informal institutions—a mechanism we incorporate by modelling the hazard of exit during periods of heightened informal bargaining (e.g., minority governments, non‑programmatic coalitions, corruption scandals). 

Has survival analyses been used in contexts like these? 

Previous applications of survival analysis to political elites reveal several patterns relevant to Peru. \citet{huber_replacing_2008} cross-national study of parliamentary democracies shows that ministerial stability depends on coalition bargaining rules rather than cabinet duration per se . In Latin America, \citet{gonzalez-bustamante_cambios_2016} find that critical events and individual traits shape minister survival during Chile’s Concertación era , while \citet{camerlo_minister_2015-1} demonstrate that cabinet reshuffles in presidential systems respond to mass protests, scandals, and the electoral calendar . Cabinet turnover also varies with executive style: \citet{carreras_presidentes_2013} shows that outsider presidents rely on inexperienced ministers, shortening tenures across Latin America . Within Chile, \citet{avendano_rotacion_2012} link high rotation to coalition bargaining costs , whereas \citet{martinez-gallardo_out_2012} identifies partisan incentives and agency risks that prompt ministers themselves to defect from presidential coalitions . Comparative surveys confirm structural drivers: \citet{fischer_duration_2012}  show that portfolio salience and regime type systematically affect ministerial durability across thirty countries. \citet{jara_iniguez_rotacion_2019} argue that high-level turnover in Ecuador stems less from managerial failures than from trust-based appointments , and \citet{mejia_es_2021} document how Colombia’s shift to coalition presidentialism lengthened cabinet survival over six decades . Together, these works highlight how institutional arrangements, critical events, and presidential strategies condition ministerial tenure—insights directly applicable to explaining Prime-Ministerial durability in Peru.



% %%%%% adding bibliography
\bibliographystyle{apacite} %%style
\renewcommand{\refname}{Bibliography}
\bibliography{Premieres} %% filename

\end{document} %% nothing after here

