\documentclass[a4paper, 12pt]{article}
\usepackage{authblk}
\renewcommand{\Affilfont}{\footnotesize}
\title{Cabinet under Pressure: Survival Analysis of Peru’s Prime Ministers Since 1980}
\author[1,2]{ Jose Manuel Magallanes \thanks{The author would like to thank the research assistants from PULSO-PUCP: Alexandra Porras, Alfredo Aro, Romina Loayza, and Ivana Delgado, for their support  and dedication to this work.}}
\affil[1]{PULSO -Institute of Social Analytics and Strategic Intelligence and Department of Social Sciences, Pontificia Universidad Catolica del Peru, San Miguel 15088, Lima, Peru}
\affil[2]{University of Massachusetts-Amherst; University of Washington -Seattle; and Universidad Nacional Mayor de San Marcos-Lima}
\affil[*]{Corresponding author: jmagallanes@pucp.edu.pe}

\usepackage{tabularx,booktabs,threeparttablex,multirow}
\usepackage{rotating} 
\date{\today}  %% manually: \date{March 6, 2025} 
\usepackage[natbibapa]{apacite} %% for bibliography
\usepackage{rotating, graphicx} %% for rotating tables
\usepackage{adjustbox} % size of plots and tables
\usepackage{chngcntr}% section numbering
\usepackage{amssymb}
\counterwithin{table}{section}\counterwithin{figure}{section}
\usepackage{Sweave}
\begin{document} % every "begin: needs and "end"
\Sconcordance{concordance:premieres.tex:premieres.Rnw:1 21 1 1 0 153 1 1 26 18 0 1 2 %
230 1 1 12 29 0 1 2 270 1}

\maketitle 
\begin{abstract}
This study examines the political durability of Peru’s Prime Ministers (Presidente del Consejo de Ministros, PCM) since the country’s democratic return in 1980. Using survival analysis, we explore how political context, institutional conditions, and crisis dynamics shape the tenure of these key presidential appointees. Through a Cox proportional hazards model, we test the influence of presidential popularity, legislative fragmentation, cabinet reshuffles, and regime instability on the risk of early dismissal or resignation. The results illuminate how informal power-sharing and institutional fragility influence executive coordination in a hyper-presidential regime.
\end{abstract}


\section*{Introduction} % * to unnumber

The median tenure of a Peruvian Prime Minister (PCM) since 1980 is 254 days (around 8 months), yet variation is striking. Ántero Flores\-Aráoz lasted only six days in November 2020, whereas Manuel Ulloa served 864 days in the early 1980s and Jorge del Castillo survived 805 days during Alan García\'s second term. Dispersion persists even within a single presidential period: three PCMs held the post during Pedro Castillo\'s first five months, while Alberto Otárola remained in office for 440 days under Dina Boluarte. This inconsistency under ostensibly similar institutional rules raises a central puzzle: Which political, institutional, and situational factors condition the durability of Peru\'s Prime Ministers?

Although the Constitution describes the PCM merely as the president’s first minister—responsible for countersigning decrees and coordinating the cabinet—practice since 1980 indicates that the office routinely performs head-of-government functions: drafting the policy agenda, negotiating confidence votes, and representing the executive in congressional interpellations. The PCM thus acts as the president’s chief political shield. During normal times, the office facilitates legislative compromise and signals technocratic competence; in crises, it functions as a “circuit-breaker,” a high-visibility scapegoat whose dismissal absorbs congressional anger and preserves presidential tenure. Analysing the determinants of PCM survival therefore illuminates how Peru’s hyper-presidential system manages accountability, blame assignment, and policy coordination in the absence of strong parties.




\section{Background}\label{backg-tables} % label for crossref

\subsection{Evolution of the Prime Minister’s formal mandate} Peru’s Prime Minister (PCM) has migrated from a cabinet chair of convenience in the 1979 Constitution to a multisector policy coordinator and modernisation czar under the post‑2000 statutory framework.  Four legal milestones define that trajectory:


\begin{itemize}
\item 1979 Constitution (arts.215–218). The PCM presides over the Council of Ministers only when the President is absent; all appointments and dismissals rest with the President, who may chair the Council at will.  The PCM’s core duty is to secure majority support inside the cabinet for bills, decree‑laws, and matters of public interest. (Constitution 1979.)

\item 1993 Constitution (arts. 122–127). Retains presidential discretion over appointments but upgrades the PCM: he/she convenes the Council, signs legislative decrees, and must present the government programme and request a vote of confidence from Congress. (Democratic Constituent Congress 1993.)
\item Organic Law of the Executive Branch – LOPE 2007 (ch. II). Charges the PCM with proposing the government’s general objectives, coordinating multisectoral national policies—especially economic and social development—and supervising entities attached to the Presidency of the Council of Ministers. (Congress of the Republic 2007, LOPE.)
\item Law 29158 (2007) on Modernisation of the State (art.19). Designates the PCM the President’s highest‑trust agent, responsible for steering public‑sector modernisation and decentralisation across all tiers of government.  (Congress of the Republic 2007, Law 29158.)
\end{itemize}

Together these reforms turned the PCM from symbolic countersignatory into the President’s principal political shield and agenda gate‑keeper. See Table \ref{tab:pcm-mandate} for details on this evolution.

\begin{sidewaystable}[p]   % ‘p’ = put on its own float page
\centering
\caption{Evolution of the PCM’s formal mandate since 1979}
\label{tab:pcm-mandate}
\renewcommand{\arraystretch}{1.2}
\begin{tabularx}{\textheight}{@{}l X X X X l@{}}  % \textheight fits the rotated width
\toprule
\textbf{Legal framework} &
\textbf{Cabinet-chair role} &
\textbf{Policy-coordination powers} &
\textbf{Appointment \& dismissal} &
\textbf{Additional statutory duties} &
\textbf{Citation}\\
\midrule
1979 Constitution &
Chairs Council only when President absent; President may chair at will &
Secures majority consent for bills and decree-laws within the cabinet &
Exclusive presidential prerogative &
None specified beyond cabinet deliberations &
Const.\ 1979, arts.\ 215–218\\[4pt]

1993 Constitution &
Convenes Council; President may still chair &
Countersigns decrees; presents programme and confidence motion to Congress &
Exclusive presidential prerogative &
Must seek confidence votes &
Const.\ 1993, arts.\ 122–127\\[4pt]

LOPE 2007 &
PCM regularly chairs Council (President rarely attends) &
Proposes government objectives; coordinates multisector policies; supervises PCM-attached entities &
Unchanged &
Broad multisector supervision &
LOPE 2007, ch.\ II\\[4pt]

Law 29158 (2007) &
As per LOPE &
Leads State modernisation and decentralisation agenda &
Unchanged &
Strategic planning; inter-governmental coordination &
Law 29158, art.\ 19\\
\bottomrule
\end{tabularx}
\end{sidewaystable}

\subsection{Informal evolution: from symbolic coordinator to political operator}

The Prime Minister’s day‑to‑day leverage has always depended less on parchment rules than on shifting presidential strategies and congressional configurations.  Three sequential informal models can be distinguished:

\begin{itemize}

\item Symbolic coordinator (1980–1990).Under Belaunde and the first García administration the PCM remained a low‑profile chair; party‑fragmented congresses preferred dealing directly with sector ministers, and premiers changed whenever coalition arithmetic shifted.  Average tenure was 405 days, but the post was still viewed as an honorific stepping‑stone rather than a power centre.

\item Gate‑keeper in an autocratic presidency (1990 – 2000). Fujimori’s autogolpe (1992) sidelined Congress and broadened decree‑law rule.  The PCM became the sole bureaucratic filter for emergency economic decrees and IMF‑negotiated reforms; loyalty trumped expertise (e.g., Hurtado Miller, Pandolfi).  Although the decade’s mean tenure fell to $\approx 302$ days, variance shrank because Fujimori kept loyalists long until a policy pivot required a full cabinet refresh.

\item Floor‑manager and lightning rod (2001 – present). With the 1993 Constitution’s confidence‑vote mechanism now routinised, the PCM took on vote counting and crisis‑absorption roles.

\begin{itemize}

\item Coalition minority years (Toledo 2001‑06): premiers (e.g., Carlos Ferrero) traded ministerial posts for legislative backing.  

\item Party‑government years (García II 2006‑11): Jorge del Castillo doubled as APRA whip, illustrating the PCM’s partisan link function.  
\item Confrontational years (PPK, Vizcarra, Castillo): Congress weaponised interpellations; presidents sacrificed PCMs (Zavala 2017; Villanueva 2018; Bellido→Vásquez→Torres 2021‑22) to defuse censure threats or reset talks before the 28 July speech.  

\item Post‑2022 caretaker presidency: Dina Boluarte retained Alberto Otárola for 440 days to project continuity amid unrest, showing the PCM as stability signal.
\end{itemize}
\end{itemize}


Across these phases, two informal prerogatives emerged:

\begin{itemize}
\item Negotiator‑in‑chief. Modern PCMs draft the annual Mensaje a la Nación, map votes for confidence motions, and lobby party spokespeople—tasks never spelled out in LOPE.

\item High‑visibility “circuit‑breaker.” Presidents now time reshuffles to absorb blame for corruption scandals (e.g., Saavedra 2016 education dispute; Chávarry 2019 judicial scandal) or economic shocks, banking on a short‑term boost in approval.

\end{itemize}

These unwritten practices explain why Prime‑Ministerial exits cluster around scandals, protests, and the July speech window—patterns the survival analysis in Section 4 explicitly models.


\subsection{Prime‑Ministerial tenure across political periods}

Peru’s Prime‑Ministerial tenure remains volatile, but volatility itself is not uniform across eras.  Using the verified dataset of 57 spells (1980–2025), Table \ref{tab:durationPerEra} summarises the distribution by political period.  The Fujimori decade exhibits the shortest average tenure (302 days) yet also the narrowest spread, reflecting the concentration of presidential power; the immediate post‑transition years average roughly 405 days, while the post‑2000 era shows the shortest median (232 days) and the widest range as presidents cycle through PCMs to manage fragmented congresses and serial crises.



\begin{table}[!h]
\centering
\caption{Duration in Days by era\label{tab:durationPerEra}}
\centering
\begin{threeparttable}
\begin{tabular}[t]{lrrrrr}
\toprule
Era & count & mean & min & max & range\\
\midrule
A.PreFujimori & 9 & 405.44 & 138 & 864 & 726\\
B.Fujimori & 13 & 302.08 & 78 & 792 & 714\\
C.PostFujimori & 35 & 256.77 & 6 & 805 & 799\\
D.before\_1980 & 167 & 262.43 & 1 & 1822 & 1821\\
\bottomrule
\end{tabular}
\begin{tablenotes}[para]
\item \textit{Source: } 
\item author calculations from official appointment resolutions and El Peruano archives.
\end{tablenotes}
\end{threeparttable}
\end{table}

\section{Literature Review}\label{sec:letrev}

The formal mandate outlined in Table \ref{tab:pcm-mandate} establishes the PCM as the president’s chief policy coordinator and political shield. The subsections that follow review comparative findings on the political, institutional, and situational determinants of ministerial tenure, providing the theoretical backdrop for our hypotheses.


\subsection{Cabinet Volatility in Peru}

Early comparative work portrays Peru as an extreme case of ministerial churn.  \citet{martinez-gallardo_out_2012} reports an average cabinet duration of just 13 months (1980–2000), less than half the Latin‑American mean and exceeded only by Brazil.  Sector‑specific studies corroborate the pattern: in the health ministry, average tenure was 13.7 months between 1935 and 2021 \citep{gozzer_duracion_2021}.  Frequent reshuffles reset reform agendas—“policies are often returned to square one” \citealp[335]{gozzer_duracion_2021} —and cascade through lower administrative layers.  Yet the causes of this volatility remain under‑specified.

\subsection{Critical Events and Shocks}

Extant research highlights critical events as proximate triggers of turnover.  \citet{dewan_corrective_2005} show that UK prime ministers reshuffle after scandals to restore popularity; \citet{camerlo_minister_2015-1} find media scandals and social protests heighten rotation in twelve Latin‑American democracies, with effects conditioned by the electoral calendar and re‑election rules.  Because Peruvian presidents cannot be re‑elected, the country fits Camerlo and Pérez‑Liñán’s scenario in which protests spike hazard early in the term, while scandals matter later \citet[616]{camerlo_minister_2015-1}. Economic crises operate similarly: \citet{gonzalez-bustamante_cambios_2016} incorporate downturns into Chilean survival models, and \citet[515]{fischer_duration_2012} argue that a minister’s post‑shock fate hinges on formal removal rules, political weight, and net benefit to the government.


\subsection{Institutional Determinants}

Turning to institutional factors, the presidential–parliamentary divide looms large.
From \citet{fischer_duration_2012} we learned that Single-party cabinets generally stay in power longer than coalitions; within coalitions, minimum-winning cabinets outlive surplus (oversized) coalitions; Cabinets endure markedly longer in parliamentary systems\footnote{Blondel’s cross-national data show presidential cabinets are the shorter-lived side of the divide \citep{blondel_government_1985}}. If the regime were authoritarian, cabinets in communist regimes lived longer than the ones from military dictatorships; and finally, Cabinets with more veto-player parties are less durable.

Some of those findings complement the ones above.  A sixty‑year study of Colombia shows that multiparty coalition presidentialism lengthens tenure, whereas two‑party periods display higher churn \citep{mejia_es_2021}.  Uruguay’s above‑average stability likewise hinges on durable coalitions; crises coincide with coalition breakdowns \citep{chasquetti_designacion_2013}.


\subsection{Ministerial Profiles}

Personal attributes cannot be ignored.  Age effects are U‑shaped, with the youngest and oldest ministers more likely to exit \citet{fischer_duration_2012}.  \citet{escobar-lemmon_coming_2010} find that being in the inaugural cabinet and holding the finance portfolio increase hazard, whereas constituency links reduce it.  Prior political experience, however, has only marginal impact \citep{escobar-lemmon_coming_2010}.  In Peru, outsider presidents such as Alberto Fujimori and Pedro Pablo Kuczynski have consistently favoured technocrats with scant partisan capital \citep{carreras_presidentes_2013,nercesian_radiografigabinetes_2019}.  Whether such technocratic cabinets enjoy longer—or shorter—survival remains an open empirical question.
% 

\subsection{Gaps in the Literature and Relevance to Peru}

Existing studies converge on three explanatory families—critical events, institutional arrangements, and individual profiles—but seldom test them jointly inside a single presidential system.  Moreover, most Latin‑American research pools multiple countries, obscuring Peru’s idiosyncrasies: (i) constitutional hyper‑presidentialism without the stabilising force of strong parties; (ii) the annual Mensaje a la Nación on 28 July, a focal point for reshuffles; and (iii) recurrent economic volatility.  By integrating LOPE’s formal mandates with comparative findings on shocks, institutions, and minister profiles, the present study builds a tailored framework for explaining why some Peruvian PCMs become expendable “circuit‑breakers” while others endure.



\section{Data and Methods}

\subsection{Dataset construction}

The event history comprises all 57 Prime‑Ministerial spells between 28 July 1980 and 8 May 2025. Start‑dates are taken from swearing‑in decrees in El Peruano; end‑dates are either the dismissal decree or the acceptance‑of‑resignation decree. Because Gustavo Adrianzén is still in office, his spell is treated as right‑censored at 8 May 2025. Each spell is split at annual cut‑points (28 July) to accommodate time‑varying covariates, following the Andersen–Gill (AG) counting‑process framework.  Under AG, each spell is broken into multiple risk intervals—one per calendar year in our design—so that covariates can vary within a subject’s tenure while preserving the Cox model’s partial‑likelihood properties.  The AG formulation treats the hazard in interval  as conditionally independent of previous intervals given the covariate history, allowing us to stack rows and estimate a single Cox regression with robust standard errors clustered on the PCM identifier.

\begin{sidewaystable}[p]
\centering
\caption{Variables, operationalisation, sources, and prior literature}
\label{tab:vars-rotated}
\renewcommand{\arraystretch}{1.2}
\begin{tabularx}{\textheight}{@{}l l X X X@{}}
\toprule
\textbf{Category} & \textbf{Variable} & \textbf{Operationalisation} & \textbf{Source} & \textbf{Previous work} \\
\midrule
Outcome 
  & Duration 
  & Days from appointment to exit or censoring 
  & Author dataset 
  & — \\ 
\addlinespace
\multirow{2}{*}{Congress} 
  & ENP 
  & Quarterly effective number of parties (seat share) 
  & JNE seat records 
  & Huber \& Martínez-Gallardo (2008); Martínez-Gallardo (2012) \\
  & Censure motion 
  & 1 if an interpellation or censure motion is tabled in a quarter 
  & Congressional Journal 
  & Camerlo \& Pérez-Liñán (2015); Dewan \& Dowding (2005) \\
\addlinespace
\multirow{2}{*}{Presidency} 
  & Approval 
  & Quarterly presidential approval (\%) 
  & Ipsos Perú surveys 
  & Samuels \& Shugart (2003); Fischer et al. (2012) \\
  & Speech window 
  & 1 if within ± 45 days of 28 July 
  & Author calculation 
  & This study (informal calendar logic) \\
\addlinespace
Economy 
  & GDP shock 
  & 1 if quarterly GDP growth < 0 
  & BCRP 
  & González-Bustamante \& Olivares (2015) \\ 
\addlinespace
Regime 
  & Fujimori 
  & 1 for 5 Apr 1992 – 28 Jul 2000 
  & Author coding 
  & Pérez-Liñán (2007); Fischer et al. (2012) \\ 
\addlinespace
Profile 
  & Technocrat 
  & 1 if PCM has never held elected office 
  & CVs \& press 
  & Carreras (2013); Escobar-Lemmon \& Taylor-Robinson (2010) \\
\bottomrule
\end{tabularx}
\end{sidewaystable}

All variables except Technocrat and Fujimori are time‑varying. The speech dummy captures the reshuffle focal point.

\subsection{Model specification}

The baseline estimation is a Cox proportional‑hazards model with shared frailty by presidential administration. Ties are handled with the Efron method, and robust standard errors are clustered on presidencies to absorb unobserved style differences.

We express the hazard for PCM spell \(i\) under president \(j\) at time \(t\) as:

\[
  h_{ij}(t \mid X_i(t), u_j)
    = h_0(t) \exp\bigl(\beta^{\mathsf{T}}X_i(t) + u_j\bigr),
  \quad u_j \sim \mathcal{N}(0,\sigma^2),
\]


\paragraph{Explanation of terms}
\begin{itemize}
  \item $h_{ij}(t \mid X_i(t), u_j)$: the hazard rate for spell $i$ under president $j$ at time $t$, i.e., the instantaneous risk of exit.
  \item $h_0(t)$: the baseline hazard function, representing the underlying risk when all covariates are zero.
  \item $X_i(t)$: the vector of (possibly time-varying) covariates for spell $i$, including ENP, approval, GDP shocks, and the speech-window dummy.
  \item $\beta$: the vector of regression coefficients quantifying each covariate’s effect on the log-hazard.
  \item $\beta^\top X_i(t)$: the linear predictor (log-relative hazard) combining covariates and coefficients.
  \item $u_j$: the shared-frailty term (a random effect) for administration $j$, capturing unobserved presidency-level heterogeneity.
  \item $u_j \sim \mathcal{N}(0,\sigma^2)$: indicates $u_j$ follows a normal distribution with mean 0 and variance $\sigma^2$, where $\sigma^2$ measures unexplained variance across presidencies.
\end{itemize}



\subsection{Diagnostics and robustness}

Global Schoenfeld test indicates no violation of proportionality.  Approval shows mild drift; a time‑interaction spline corrects it.

Piece‑wise exponential and Weibull baselines reproduce sign and significance of all coefficients.

Competing‑risks model (forced dismissal vs. voluntary resignation) shows ENP and the speech window mainly drive forced exits.

Influence diagnostics (dfbetas) show no spell exerts undue leverage; dropping the six‑day Bellido spell leaves estimates unchanged.

3.5 Software and replication

Models are estimated in R 4.3 using packages \emph{survival}, \emph{coxme}, and \emph{ggsurvfit}. All data and scripts are archived on Zenodo.


% %%%%% adding bibliography
\bibliographystyle{apacite} %%style
\renewcommand{\refname}{Bibliography}
\bibliography{Premieres} %% filename

\end{document} %% nothing after here

